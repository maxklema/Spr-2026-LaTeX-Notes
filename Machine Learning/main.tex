der line width

\rhead{\today}
\lhead{\textbf{Maxwell Klema}}
\cfoot{\thepage}


\title{Machine Learning}
\date{Spring 2026, Purdue University Fort Wayne}
\author{Maxwell Klema | Professor Mehrdad Hajiarbabi}

\lstset{
    frame=single, % Adds a single line frame around the code
}

\begin{document}

\maketitle
\pagestyle{fancy}

\vspace{11pt}
\section{Lecture 1 - Machine Learning : Learning Methodology}
\vspace{11pt}

"The best way to learn is to teach" - Frank Oppenheimer

\vspace{11pt}
\subsection{Learning Methodology}
\vspace{11pt}

It is important to learn from examples. Machine learning is all about "learning from the data". It is also important to compare with something you are familiar with. For example, when picking up C language after knowing anoth\documentclass[11pt]{article}
\usepackage[a4paper,left=2.5cm,right=2.5cm,top=\dimexpr15mm+1.5\baselineskip,bottom=2cm]{geometry}
\usepackage[english]{babel}
\usepackage[T1]{fontenc}
\usepackage[utf8]{inputenc}
\usepackage{geometry}
\usepackage{fancyhdr} % Headings
\usepackage{listings} % Code
\usepackage{parskip} % Spacing for Paragraphs
\usepackage{mdframed} % Asides
\usepackage{caption}
\usepackage{graphicx}
\usepackage{amsmath}
\usepackage{hyperref}

% Asides
\newenvironment{aside}
  {\begin{mdframed}[style=0,%
      leftline=false,rightline=false,leftmargin=2em,rightmargin=2em,%
          innerleftmargin=0pt,innerrightmargin=0pt,linewidth=0.75pt,%
      skipabove=7pt,skipbelow=7pt]\small}
  {\end{mdframed}}

\renewcommand{\headrulewidth}{.2mm} % heer language, like Java, or C\#, or C++, it is easier than learning it as your first language. In deep learning, transfer learning is when you build a model for one task and then use it for another task. It can be expensive, takes a lot of money, but once you have one model for a certain task, you can use it for another task using transfer learning.

It is also important to get your "hands dirty" and try out new projects through trial and error. In machine learning, this is known as reinforcement learning - try to get the agent to do something, and if it fails, give it a penalty to try something else until it "succeeds" or "learns".

Reading books or papers can be very helpful. Same thing with university lectures, too. 

Additionally, you should work with someone who is smarter than you, someone you can look up to and learn from who has been in your shoes before. 

Throughout the whole learning process, it is important to reflect and take notes: what are mistakes to avoid in the future? What are good practices to keep? Applying AI tools can help with the process, too, as long as you use them with caution.

\vspace{11pt}
\textbf{In General, there are two learning methods}:
\begin{itemize}
    \item Learning by instruction - Have a detailed example and instruction you can follow to learn the basics.
    \item Learning by discovery - important to pivot to eventually, no clear direction or example to follow through.
\end{itemize}
\vspace{11pt}

In Software development, you must have the ability to find the answer to a question/problem/issue quickly. Similarly, it is critical to have the ability to learn how to find the answer quickly and the ability to ask good questions.

\vspace{11pt}
\textbf{Learning-ability is very important!}
\vspace{11pt}

For students in introductory machine learning classes, to become successful, you should:
\begin{itemize}
    \item Attend the class
    \item Join the discussion
    \item Review course materials in time (15-30 minutes post-lecture)
    \item Understand course materials and homework
    \item Work on homework problems
    \item Work on course presentation
    \item Work on course project
    \item Practice! Practice! Practice!
\end{itemize}

\vspace{11pt}
\textbf{The 80/20 Rule}
\vspace{11pt}

A small portion of people (20\%) may have the biggest effect (80\%). In this class, it means that 20\% of the technique takes 80\% of the time. In this class, we want to focus on the 20\%. What about the other 80\%, we can use the learning methods that we just discussed about: textbook, asking the professor, AI tools, lectures, projects, etc.

\vspace{11pt}
\textbf{AI Tools Policy}
\vspace{11pt}

AI tools include models such as ChatGPT or Gemini. It is good to use AI tools to learn course materials. However, for homework assignments, it is important to not use AI tools for code generation or answer questions (except if the assignment allows it). But, you can ask AI tools or Google on the error messages or debugging the code.

For the course project, feel free to use any AI tools. But, again, use them in a wise-way and be careful. They can get very buggy very fast.

Do not just take everything from AI-generated code without understanding every line of code. Make sure that you really need that line of code. You also need to test it.

If code is generated by AI tools, indicate this in the comment section of the program (which AI tool + prompt).

\vspace{11pt}
\subsection{Google Colab}
\vspace{11pt}

In Google Colab, you can upload any Jupyter Notebook from Google Driver (.ipynb or .py), GitHub, Local SSD, etc. You can also create an empty notebook.

\vspace{11pt}
\begin{aside}
    It is important to export all notebooks as both .py and .ipynb as in peer reviews, it is easier to review code without the other parts of the notebook (text).
\end{aside}
\vspace{11pt}

In Jupyter Notebook, you can utilize the Markdown Language (\# is heading, \#\# is sub-heading, etc.).

You can also separate the notebook by different sections for better organization. Each section can contain codeblock cells. You are able to run one cell at a time, run all cells, or run the current cell and all cells before.

There is an option to change the runtime. You can choose the runtime type (Python, R, Julia), the hardware accelerator (TPU, CPU, GPU). Google Colab is very good about updating their software with latest hardware accelerators and runtime versions.

On the left-hand side (LHS) of Google Colab, you can find and replace, see table of contents, filter code snippets, and manage keys. They keys can be used for APIs (such as OpenAI or HuggingFace APIs). There is also a file system.

\vspace{11pt}
\begin{aside}
    Do not store files in the file system without backing them up locally, or storing them elsewhere in your Google Drive. When you remove the runtime, the files are also cleared, and you may lose progress.
\end{aside}
\vspace{11pt}

In the code block sections, you can execute special UNIX commands that follow a ! symbol. For example, if using a NVIDIA GPU, you can use !nividia-smi to get information about the GPU.

\vspace{11pt}
\textbf{More Examples}:
\vspace{11pt}

\vspace{11pt}
\begin{lstlisting}[language=Python, breaklines=true]
import tensorflow as tf
tf.config.list_physical_devices('GPU')    
\end{lstlisting}
\vspace{11pt}

\vspace{11pt}
\begin{lstlisting}[language=Python, breaklines=true]
import torch
torch.cuda.is_available() # => True  
\end{lstlisting}
\vspace{11pt}

\vspace{11pt}
\begin{lstlisting}[language=Python, breaklines=true]
# Using Google Drive
from google.colab import drive
drive.mount('/content/drive')
# => Mounted at /content/drive
\end{lstlisting}
\vspace{11pt}

\vspace{11pt}
\begin{lstlisting}[language=Python, breaklines=true]
!pwd
!ls

# => /content
# => drive sample_data

!cd drive/MyDrive
!ls

# => drive sample_data
\end{lstlisting}
\vspace{11pt}

Using the '!' only works for that one command and then closed the environment. To make it persist, use the \% symbol.

\vspace{11pt}
\begin{lstlisting}[language=Python, breaklines=true]
%cd drive/MyDrive
!ls

# => Contents of drive/MyDrive
\end{lstlisting}
\vspace{11pt}

If the Google Colab sessions keeps timing out, you can use a special connectButton() function in DevTools that keeps pressing the connect button repeatedly to keep the session awake. Another method is using a GPU provided by Kaggle.

\vspace{11pt}
\section{Lecture 2 - Python in Machine Learning}
\vspace{11pt}

\textbf{Deep Learning} is a spacial-type of model for learning. It can use its own knowledge base to learn and recognize patterns.

\textbf{Review}: In Google Collab, '!' opens a session, runs the command, and then closes the session. Meanwhile, the \% is the 'magic command' begins it keeps a session open after running a single command. '!' is mostly used for pip installing different packages for different libraries and such.

Python is ranked as the \#2 top language in both GitHub (By number of projects) as well as Stack Overflow (By number of question tags).

\vspace{11pt}
\textbf{How to Learn a New Programming Language Fast?}
\vspace{11pt}

\textbf{Compare the language to another language that you know}, build projects, look at pre-existing projects on Github, look at examples, watch tutorials, \textbf{Apply ChatGPT to learn}, \textbf{pay attention to langauge features \& topics}

\vspace{11pt}
\textbf{Differences from C Programming Language}
\vspace{11pt}

\begin{itemize}
  \item C is strongly typed; Python is weakly typed.
  \item Python is historically interpreted; C is compiled.
  \item \textbf{No main function in Python.}
  \item C requires manual memory management; Python manages memory for you.
  \item \textbf{Python uses indentation; C uses curly braces.}
  \item \textbf{C uses semicolons; Python does not.}
  \item \textbf{C is procedural (not OOP); Python is OOP.}
  \item Python reads more like English than C.
  \item \textbf{There are no "do ... while" loops in Python}
  \item \textbf{Variable scope within a function: Python variable scope is local to a function (while/for/if statements do not create a new scope) while C local to a block.}
  \item \textbf{Exception handling: In C, everytime you call a function, you check some variable and return a error code. In Python, you can use try and except.}
  \item \textbf{Passing an object in Python is similar to pass by reference, while passing in an int, float, etc is pass by variable (value).}
\end{itemize}

\vspace{11pt}
\textbf{Language Features or Topics}
\vspace{11pt}

The paradigm of a language is important to consider. Is it OOP, procedural, functional, etc? Additionally, are variables weak/strong typed, static or dynamic? For conditional branches, how do if/else, while, and for statements work?

For functions, are these pass-by-value, pass-by-pointer, etc? How do return types work, what about recursive functions, default parameters, etc?

What are popular libraries in the language? 

What are the build-in data types (int, double, char, string, etc.) What about advanced data types and structures (struct, pointers, list, tuple, dictionary, etc.).

What about I/O (file, socket, etc), memory management, and debugging and the IDE?

\vspace{11pt}
\textbf{Python Data Types}
\vspace{11pt}

\begin{itemize}
  \item Boolean
  \item Numbers (\texttt{int}, \texttt{float})
  \item String
  \item Range (sequence of numbers)
  \item List (array — sequence of mutable values using \texttt{[ ... ]})
  \item Tuple (immutable list using parentheses; useful for returning multiple values)
  \item Dictionary (key:value pairs; similar to a JavaScript Object/JSON; uses \{ \})
  \item Set (collection of unique values; uses \{ \})
\end{itemize}


\vspace{11pt}
\textbf{Python Examples}
\vspace{11pt}


\begin{lstlisting}[language=Python, breaklines=true]
myvar = 123
print(myvar) # => 123
myvar = "Hello World"
print(myvar) # => "Hello World"
\end{lstlisting}
\vspace{11pt}
\begin{lstlisting}[language=Python, breaklines=true]
for i in range(4):
    #print("#")
    print("#", end="")

print()
print("#"*4)

# => ####
# => ####
\end{lstlisting}
\vspace{11pt}
\begin{lstlisting}[language=Python, breaklines=true]
total = 0 # do NOT use "sum" for variable name!!!
for i in range(11):
    total += i
    #print("sum = %d" % (sum)) => more C style
print(f"sum = {total}")

# => sum of first 10 numbers
\end{lstlisting}
\vspace{11pt}
\begin{lstlisting}[language=Python, breaklines=true]
def calculate_sum(n):
    sum = 0
    for i in range (n):
        sum += i
    return sum
\end{lstlisting}
\vspace{11pt}
\begin{lstlisting}[language=Python, breaklines=true]
def is_even(n):
    return (n % 2 == 0)

for i in range(11):
    if is_even(i):
        print(f"{i} is even")
    else:
        print(f"{i} is odd")
\end{lstlisting}
\vspace{11pt}
\begin{lstlisting}[language=Python, breaklines=true]
x = int(input("x: "))
y = int(input("y: "))
print(x+y)
print(x/y)
print(x//y) # floor division

z = x/y
print(f"{z:.50f}") # formats the floating-point value z 
# with fixed-point notation to 50 digits after the decimal
\end{lstlisting}
\vspace{11pt}
This code is not robust: What if the user enters a non-numeric input or y = 0? The program can use a try..except to ensure that data types are valid.
\vspace{11pt}
\begin{lstlisting}[language=Python, breaklines=true]
people = {
    "Chen": "12343243",
    "David": "43243243"
}

name = "Chen"
if name in peopple:
    print(f"Number: {people[name]}")
\end{lstlisting}

\begin{lstlisting}[language=Python, breaklines=true]
import csv

# Upload phonebook.csv file

name = "Chen"
number = "324234234"
'''
file = open("phonebook.csv", "a")
writer = csv.writer(file)
writer.writerow([name, number])
...
'''
\end{lstlisting}
\vspace{11pt}
\begin{lstlisting}[language=Python, breaklines=true]
#Blurs an image

from PIL import Image, ImageFilter

#Upload "bridge.bmp" file

#blur image
before = Image.open("Bridge.bmp")
after = before.filter(ImageFilter.BoxBlur(5))
after.save("out1.bmp")
\end{lstlisting}
\vspace{11pt}
\begin{lstlisting}[language=Python, breaklines=true]
#zip() function
nums = [1,2,3]
strings = ["hi", "cs590", "DL"]

for i, string in zip(nums, strings):
    print(i, string)

# Go through two lists and get the corresponding elements one by one
# 1 hi
# 2 cs590
# 3 DL

# zip() prints the min(length) of both lists in 
# case one is larger than the other.
\end{lstlisting}

\vspace{11pt}
\begin{lstlisting}[language=Python, breaklines=true]
# enumerate() function
for index, string in enumerate(strings):
    print(index, string)

# enumerate() will give you the index of each element in the list.
\end{lstlisting}

\vspace{11pt}
\begin{lstlisting}[language=Python, breaklines=true]
#map() function

numbers = (1,2,3,4)
result = map(lambda: x: x**2, numbers)
print(list(result))

# [1, 4, 9, 16]
\end{lstlisting}
\vspace{11pt}
\begin{lstlisting}[language=Python, breaklines=true]
# List comprehensions

squares = [n**2 for n in range(1,5)]
print(squares)

# Using the SQL analogy
# you could think of SELECT, FROM, WHERE, etc..
\end{lstlisting}
\vspace{11pt}
Slicing

\begin{lstlisting}[language=Python, breaklines=true]
org_data = list(range(1,11))
print(org_data) # prints 1 to 10
print(org_data[2:5]) # prints index 2 through 4
print(org_data[:5]) # first 5 index
print(org_data[5:]) # 6th index to the end
print(org_data[-1]) # last element
print(org_data[-2]) # second from the last
print(org_data[4:-4]) # [5,6]
\end{lstlisting}

\vspace{11pt}
\section{Lecture 3 - ChatGPT APIs}
\vspace{11pt}

From the last lecture, we talked about basic Python concepts. What are some ways to pick up Python quickly? We can compare it to other languages, build projects, use AI, etc.

Let's look into Python Libraries

\vspace{11pt}
\begin{lstlisting}[language=Python, breaklines=true]
import numpy as np
a = [1,2,3]
b = [4,5,6]
c = np.maximum(a,b)
print(c)
\end{lstlisting}
\vspace{11pt}

\vspace{11pt}
\begin{lstlisting}[language=Python, breaklines=true]
from numpy import maximum
c = maximum(a,b)
print(c)

# It is better to import np if you are going to use multiple different numpy functions
\end{lstlisting}
\vspace{11pt}

Note: when we import ...

\vspace{11pt}
\begin{lstlisting}[language=Python, breaklines=true]
x = int(input("x: ")) #error prone
print(x)
\end{lstlisting}
\vspace{11pt}

\vspace{11pt}
\begin{lstlisting}[language=Python, breaklines=true]
while True:
    try:
        x = int(input("x: ")) #error prone
    except ValueError:
        print("That is not an int!")
        continue
break
print(x)
\end{lstlisting}
\vspace{11pt}

\newpage
\textbf{Audio Example}

\vspace{11pt}
\begin{lstlisting}[language=Python, breaklines=true]
!pip install gtts

from gtts import gTTS
from IPython.display import Audio
tts = gTTS('Hello cs492 machine learning!')
tts.save('1.wav')
sound_file = '1.wav'
Audio(sound_file, autoplay=True)
\end{lstlisting}
\vspace{11pt}

\subsection{ChatGPT APIs}
\vspace{11pt}

\subsubsection{Motivation}
\vspace{11pt}

In the browser, we use the ChatGPT model. Every question is sent to the ChatGPT model behind-the-scenes and then responds back to the user session in the browser.

Say you want to create your own application using ChatGPT. For example, we want to create an application that adds appropriate emojis to any given input, and you want to power it by ChatGPT.

Instead of doing this in the ChatGPT app, we may want to do this in our own website. We need to use the ChatGPT API for this, but it can be much slower and more expensive.

The \textbf{ChatGPT API} (Application Programming Interface) enables programmatic interaction with the ChatGPT model. You tell ChatGPT what you want it to do in a programmatic way, then ChatGPT returns the response by following what you ask it to do.

\vspace{11pt}
\subsubsection{Setup}
\vspace{11pt}

Go to \href{https://www.openai.com/api}{https://www.openai.com/api}. Then, sign in with your Google account. You will probably have to add at least \$5 in credits to use the API.

After you put money in, you can go to home > API Keys > Create new secret key. You will use this secret key in your codebase.

Example:

\vspace{11pt}
\begin{lstlisting}[language=Python, breaklines=true]
!pip install langchain_openai

from langchain_openai import OpenAI

llm = OpenAI()

question = ["Are you working propertly?"]
response = llm.invoke(question)

# key is already added before as a environment secret

print(response)
\end{lstlisting}
\vspace{11pt}

It is important to use Google Colab secrets and place your OpenAI API Key there.

\vspace{11pt}
\begin{lstlisting}[language=Python, breaklines=true]
!pip install openai
!pip install gradio #Python GUI (similar to HTML)

import openai
import gradio as gr
import json
from typing import List, Dict, Tuple

client = openai.OpenAI()

try:
    response = client.chat.completions.create(
        model="gpt-3.5-turbo",
        messages = [{'role':'user', 'content': "test"}],
        max_tokens=1, #max output tokens (~ 1 word)
    )
    print("Set ChatGPT API successfully!!")
except:
    print("There seems to be some sort of error")
\end{lstlisting}
\vspace{11pt}

In many AI models, a higher \textbf{temperature} means a more "randomized" output, whereas a lower temperature means a more predictable output.

The following demo uses gradio as a GUI library and Google's genai library to create a grammar fixer:

\vspace{11pt}
\begin{lstlisting}[language=Python, breaklines=true]
from google import genai
from google.colab import userdata
from google.genai.types import GenerateContentConfig
import gradio as gr # Moved import here to be consistent with other imports

client = genai.Client(api_key=userdata.get('GEMINI_API_KEY'))
prompt_for_summarization = "Please correct the following sentence with proper spelling, grammer, and punctuation."

# function to reset the conversation
def reset() -> list:
    return []

# function to call the model to generate
def interact_summarization(prompt: str, article: str, temp = 1.0) -> list[tuple[str, str]]:
    '''
    * Arguments

      - prompt: the prompt that we use in this section

      - article: the article to be summarized

      - temp: the temperature parameter of this model. Temperature is used to control the output of the chatbot.
              The higher the temperature is, the more creative response you will get.

    '''
    input = f"{prompt}\n{article}"

    response = client.models.generate_content(
      model="gemini-3-flash-preview",
      contents=input,
      config=GenerateContentConfig(
        temperature=temp,        # creativity / randomness
        max_output_tokens=2000,  # cap on response length
      ),
    )

    # Check if the response has choices and if the first choice has a message and content
    if response.candidates and response.candidates[0].content and response.candidates[0].content.parts:
        return [(input, response.candidates[0].content.parts[0].text)]
    else:
        return [(input, "No summary could be generated.")]

# this part generates the Gradio UI interface
with gr.Blocks() as demo:
    gr.Markdown("# Part1: Summarization\nFill in any article you like and let the chatbot summarize it for you!!")
    chatbot = gr.Chatbot()
    prompt_textbox = gr.Textbox(label="Prompt", value=prompt_for_summarization, visible=True)
    article_textbox = gr.Textbox(label="Article", interactive = True, value = "helo hw r u?")
    with gr.Column():
        gr.Markdown("#  Temperature\n Temperature is used to control the output of the chatbot. The higher the temperature is, the more creative response you will get.")
        temperature_slider = gr.Slider(0.0, 2.0, 1.0, step = 0.1, label="Temperature")
    with gr.Row():
        sent_button = gr.Button(value="Send")
        reset_button = gr.Button(value="Reset")

    sent_button.click(interact_summarization, inputs=[prompt_textbox, article_textbox, temperature_slider], outputs=[chatbot])
    reset_button.click(reset, outputs=[chatbot])

demo.launch(debug = True)
\end{lstlisting}
\vspace{11pt}







\end{document}
