\documentclass[11pt]{article}
\usepackage[a4paper,left=2.5cm,right=2.5cm,top=\dimexpr15mm+1.5\baselineskip,bottom=2cm]{geometry}
\usepackage[english]{babel}
\usepackage[T1]{fontenc}
\usepackage[utf8]{inputenc}
\usepackage{geometry}
\usepackage{fancyhdr} % Headings
\usepackage{listings} % Code
\usepackage{parskip} % Spacing for Paragraphs
\usepackage{mdframed} % Asides
\usepackage{caption}
\usepackage{graphicx}
\usepackage{amsmath}

% Asides
\newenvironment{aside}
  {\begin{mdframed}[style=0,%
      leftline=false,rightline=false,leftmargin=2em,rightmargin=2em,%
          innerleftmargin=0pt,innerrightmargin=0pt,linewidth=0.75pt,%
      skipabove=7pt,skipbelow=7pt]\small}
  {\end{mdframed}}

\renewcommand{\headrulewidth}{.2mm} % header line width

\rhead{\today}
\lhead{\textbf{Maxwell Klema}}
\cfoot{\thepage}


\title{Machine Learning}
\date{Spring 2026, Purdue University Fort Wayne}
\author{Maxwell Klema | Professor Mehrdad Hajiarbabi}

\lstset{
    frame=single, % Adds a single line frame around the code
}

\begin{document}

\maketitle
\pagestyle{fancy}

\vspace{11pt}
\section{Lecture 1 - Machine Learning : Learning Methodology}
\vspace{11pt}

"The best way to learn is to teach" - Frank Oppenheimer

\vspace{11pt}
\subsection{Learning Methodology}
\vspace{11pt}

It is important to learn from examples. Machine learning is all about "learning from the data". It is also important to compare with something you are familiar with. For example, when picking up C language after knowing another language, like Java, or C\#, or C++, it is easier than learning it as your first language. In deep learning, transfer learning is when you build a model for one task and then use it for another task. It can be expensive, takes a lot of money, but once you have one model for a certain task, you can use it for another task using transfer learning.

It is also important to get your "hands dirty" and try out new projects through trial and error. In machine learning, this is known as reinforcement learning - try to get the agent to do something, and if it fails, give it a penalty to try something else until it "succeeds" or "learns".

Reading books or papers can be very helpful. Same thing with university lectures, too. 

Additionally, you should work with someone who is smarter than you, someone you can look up to and learn from who has been in your shoes before. 

Throughout the whole learning process, it is important to reflect and take notes: what are mistakes to avoid in the future? What are good practices to keep? Applying AI tools can help with the process, too, as long as you use them with caution.

\vspace{11pt}
\textbf{In General, there are two learning methods}:
\begin{itemize}
    \item Learning by instruction - Have a detailed example and instruction you can follow to learn the basics.
    \item Learning by discovery - important to pivot to eventually, no clear direction or example to follow through.
\end{itemize}
\vspace{11pt}

In Software development, you must have the ability to find the answer to a question/problem/issue quickly. Similarly, it is critical to have the ability to learn how to find the answer quickly and the ability to ask good questions.

\vspace{11pt}
\textbf{Learning-ability is very important!}
\vspace{11pt}

For students in introductory machine learning classes, to become successful, you should:
\begin{itemize}
    \item Attend the class
    \item Join the discussion
    \item Review course materials in time (15-30 minutes post-lecture)
    \item Understand course materials and homework
    \item Work on homework problems
    \item Work on course presentation
    \item Work on course project
    \item Practice! Practice! Practice!
\end{itemize}

\vspace{11pt}
\textbf{The 80/20 Rule}
\vspace{11pt}

A small portion of people (20\%) may have the biggest effect (80\%). In this class, it means that 20\% of the technique takes 80\% of the time. In this class, we want to focus on the 20\%. What about the other 80\%, we can use the learning methods that we just discussed about: textbook, asking the professor, AI tools, lectures, projects, etc.

\vspace{11pt}
\textbf{AI Tools Policy}
\vspace{11pt}

AI tools include models such as ChatGPT or Gemini. It is good to use AI tools to learn course materials. However, for homework assignments, it is important to not use AI tools for code generation or answer questions (except if the assignment allows it). But, you can ask AI tools or Google on the error messages or debugging the code.

For the course project, feel free to use any AI tools. But, again, use them in a wise-way and be careful. They can get very buggy very fast.

Do not just take everything from AI-generated code without understanding every line of code. Make sure that you really need that line of code. You also need to test it.

If code is generated by AI tools, indicate this in the comment section of the program (which AI tool + prompt).

\vspace{11pt}
\subsection{Google Colab}
\vspace{11pt}

In Google Colab, you can upload any Jupyter Notebook from Google Driver (.ipynb or .py), GitHub, Local SSD, etc. You can also create an empty notebook.

\vspace{11pt}
\begin{aside}
    It is important to export all notebooks as both .py and .ipynb as in peer reviews, it is easier to review code without the other parts of the notebook (text).
\end{aside}
\vspace{11pt}

In Jupyter Notebook, you can utilize the Markdown Language (\# is heading, \#\# is sub-heading, etc.).

You can also separate the notebook by different sections for better organization. Each section can contain codeblock cells. You are able to run one cell at a time, run all cells, or run the current cell and all cells before.

There is an option to change the runtime. You can choose the runtime type (Python, R, Julia), the hardware accelerator (TPU, CPU, GPU). Google Colab is very good about updating their software with latest hardware accelerators and runtime versions.

On the left-hand side (LHS) of Google Colab, you can find and replace, see table of contents, filter code snippets, and manage keys. They keys can be used for APIs (such as OpenAI or HuggingFace APIs). There is also a file system.

\vspace{11pt}
\begin{aside}
    Do not store files in the file system without backing them up locally, or storing them elsewhere in your Google Drive. When you remove the runtime, the files are also cleared, and you may lose progress.
\end{aside}
\vspace{11pt}

In the code block sections, you can execute special UNIX commands that follow a ! symbol. For example, if using a NVIDIA GPU, you can use !nividia-smi to get information about the GPU.

\vspace{11pt}
\textbf{More Examples}:
\vspace{11pt}

\vspace{11pt}
\begin{lstlisting}[language=Python]
import tensorflow as tf
tf.config.list_physical_devices('GPU')    
\end{lstlisting}
\vspace{11pt}

\vspace{11pt}
\begin{lstlisting}[language=Python]
import torch
torch.cuda.is_available() # => True  
\end{lstlisting}
\vspace{11pt}

\vspace{11pt}
\begin{lstlisting}[language=Python]
# Using Google Drive
from google.colab import drive
drive.mount('/content/drive')
# => Mounted at /content/drive
\end{lstlisting}
\vspace{11pt}

\vspace{11pt}
\begin{lstlisting}[language=Python]
!pwd
!ls

# => /content
# => drive sample_data

!cd drive/MyDrive
!ls

# => drive sample_data
\end{lstlisting}
\vspace{11pt}

Using the '!' only works for that one command and then closed the environment. To make it persist, use the \% symbol.

\vspace{11pt}
\begin{lstlisting}[language=Python]
%cd drive/MyDrive
!ls

# => Contents of drive/MyDrive
\end{lstlisting}
\vspace{11pt}

If the Google Colab sessions keeps timing out, you can use a special connectButton() function in DevTools that keeps pressing the connect button repeatedly to keep the session awake. Another method is using a GPU provided by Kaggle.



























\end{document}
